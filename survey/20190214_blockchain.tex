\documentclass[twocolumn, uplatex]{jarticle}

\usepackage[top=10truemm,bottom=30truemm,left=25truemm,right=25truemm]{geometry}
\title{ブロックチェーンの暗号通貨以外の応用事例について}
\date{2019年2月14日}
\author{東京電機大学 情報セキュリティ研究室 学部3年\\庄司幸平}

\begin{document}
	\maketitle

	\section{はじめに}
	2017年では仮想通貨元年とよばれ、暗号通貨が話題に上がった。暗号通貨の中核をなす技術にブロックチェーンがある。暗号通貨に用いられるブロックチェーンをブロックチェーン1.0と呼ぶのに対して、金融関係に用いられるブロックチェーンをブロックチェーン2.0、非金融分野において用いられるものをブロックチェーン3.0と呼ばれている。本メモでは、ブロックチェーン3.0に着目をし、ブロックチェーンの暗号通貨以外での応用例について調べてまとめる。

	\section{調査事項}

	\subsection{商品販売システム}
	オフィスグリコのようなサービスにおける無人型販売形式では、ある一定期間ごとに販売者が在庫の確認と売上実績の集計、及び在庫の補填を行っている。
	DICOMO2017シンポジウムにて安井らの「ブロックチェーンを用いた持続可能な商品販売システムの検討」では、Ethereumプラットフォーム上に構築されたスマートコントラクトを用いて、商品取引の記録及び在庫管理を自動化している。
	これにより
	販売者の責務を在庫を維持するための商品の追加に限定することができ、商品販売における高い持続性の実現を可能としている。
	
	\subsection{ログストレージとしてのブロックチェーン}
	普段我々が使用しているコンピュータでは、ログと呼ばれる記録が存在している。シビラ株式会社では、ログストレージとしてブロックチェーンを利用したサービスを理リリースしている。ブロックチェーンの「改ざんできない」という性質を活かし、ログ情報をブロックチェーンに書き込み、誰も変更・修正できないようにするサービスである。これにより問題発生時に原因の救命や責任の所在を明確にするだけでなく、悪意を持ったユーザによるログデータの改ざんを防ぐことができる。
	
	\section{考察}
	ブロックチェーンのメリットとして、一つ前のブロックのハッシュ値を用いることで改ざんを困難にしている点と、それらを分散的に記録するためシステムダウンに強いということが挙げられる。これらを情報セキュリティの3大要件と比較してみると、改ざん困難性から完全性が、分散的に保存されているためデータが飛んで消えてしまう可能性が低いため可用性が、ブロックチェーンを用いた分散台帳技術によって高く保たれると考える。
	
	\section{まとめ}
	暗号通貨から始まったブロックチェーン及び分散台帳技術は、いまや様々な分野に応用されようとしている。ブロックチェーンを用いた分散台帳技術は情報セキュリティのCIAのうち可用性と完全性は高いため、その特性を活かした応用分野への発展を期待している。
	
	\section{参考文献}
	\begin{itemize}
	\item DICOMO2017/ブロックチェーンを用いた持続可能な商品販売システムの検討
	\item https://sivira.co/pr/press/20160819-01-ja.html
	\end{itemize}
\end{document}